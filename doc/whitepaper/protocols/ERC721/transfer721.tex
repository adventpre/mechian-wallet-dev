\subsection{Transfer}
\label{sec:721Transfer}

\sectlof
We continue with the notation and indices from the `\hyperref[sec:Mint]{Mint}' section.\\
\\
Suppose Alice wishes to transfer ownership of the ERC-721 token with tokenId `$\alpha$' to Bob, but under zero-knowledge.\\
\\
In the `Mint' section, we saw how Alice can create an `ERC-721 commitment' $Z_\alpha$ within the Shield contract which:
\begin{itemize}
  \item hides an underlying ERC-721 token with tokenId `$\alpha$'; and
  \item hides and binds Alice as the owner of $Z_\alpha$ (and hence of $\alpha$) through an ownership keypair $(sk_A^Z, pk_A^Z)$.
\end{itemize}
\ \\
Recall our privacy intentions:\\
Alice wishes to be able to transfer ownership of an ERC-721 token under zero-knowledge, so that the following become private:
	\begin{center}
		\begin{framed}
      \begin{enumerate}
        \item All details of the ERC-721 token (the `asset').
        \item The identity of the sender of the token (`Alice').
        \item The identity of the receiver of the token.
      \end{enumerate}
    \end{framed}
  \end{center}
Recall that minting a token commitment does not yet afford Alice any privacy (see the warning in Figure~\ref{fig:nfMintWarning}). Only with subsequent transfers will the whereabouts of $\alpha$ and the owner of $\alpha$ be hidden.

For Alice to transfer ownership of $\alpha$ within the Shield contract, under zero knowledge, she follows the steps in Figure~\ref{fig:nfTransferAlgorithm}.

\begin{figure}[htbp]
  \ContinuedFloat*
	\begin{center}
		\begin{framed}
      \begin{tabular}{p{16cm}}	
       \textbf{ Non-fungible transfer algorithm} \\
        \\
        \midrule
        \textbf{Bob's steps:}\\
        \begin{enumerate}
          \item Before Alice can send him anything, Bob must register his public keys $pk_B^Z$ and $pk_B^W$ against both his public Ethereum address $pk_B^E$ and his unique name `Bob' within the \hyperref[sec:pkd]{PKD}.
          \setcounter{ongoingEnumCounter}{\value{enumi}}
        \end{enumerate}
        \ \\
        \midrule
        \textbf{Alice's steps:}\\
        \begin{enumerate}
          \setcounter{enumi}{\value{ongoingEnumCounter}}
          \item Generate a random salt $\sigma_{\vec{AB}}$.
          \item Lookup Bob's `zkp' public key $pk_B^Z$ from the PKD.
          \item Compute $Z_B := h(\;\alpha\;|\;pk^Z_B\;|\;\sigma_{\vec{AB}}\;)$, a token commitment which represents $\alpha$.
          \item Compute $N_A := h(\;\sigma_{A}\;|\;sk^Z_A\;)$, the nullifier of Alice's commitment $Z_A$.
          \item Get $\psi_{Z_A}$ -- the sister-path of $Z_A$ -- from the Shield contract (see Details below).
          \item Get the latest Merkle root from the Shield contract: $\roott_{n+m-1}$ (see Details below).
          \item Set public inputs $x = (\;N_A,\;\roott_{n+m-1},\;Z_B)$
          \item Set private inputs $\omega = (\alpha,\;\psi_{Z_A},\;sk_A,\;\sigma_{A},\;pk_B,\;\sigma_{\vec{AB}})$
          \item Select $C_{nft-transfer}(\;\omega,\;x\;)$ -- the set of constraints which are satisfied if and only if:
          \begin{enumerate}
            \item $pk_A$ equals $h(\;sk_A\;)$; (Proof of knowledge of the secret key to $pk_A$) (see Details for why $pk_A$ isn't an input to $C$)
            \item $Z_A$ equals $h(\;\alpha\;|\;pk_A\;|\;\sigma_A\;)$ (Proof of the constituent values of $Z_A$) (see Details for why $Z_A$ isn't an input to $C$)
            \item $\roott_{n+m-1}$ equals $h\br*{\psi_{1}\;|...|\;h\br*{\psi_{d-2}\;|\;h\br*{\psi_{d-1}\;|\;Z_A}\;}...}$ (Proof that $Z_A$ belongs to the on-chain Merkle Tree)
            \item $N_A$ equals $h(\;\sigma_{A}\;|\;sk^Z_A\;)$ (Proof $N_A$ is indeed the nullifier of $Z_A$)
            \item $Z_B$ equals $h(\;\alpha\;|\;pk^Z_B\;|\;\sigma_{\vec{AB}}\;)$ (Proof that $Z_B$ contains the same asset as $Z_A$)
          \end{enumerate}
          \item Generate $\pi := P(\;p_C\;,\;x,\;\omega\;)$; a proof of knowledge of satisfying arguments $(\omega, x)\;s.t.\;C(\omega, x) = 1$. Recall: $p_C$ -- the proving key for $C$ -- will be stored on Alice's computer.
           
          The pair $(\pi, x)$ is the zk-SNARK which attests to knowledge of private inputs $\omega$ without revealing them.
          \item Send $(\pi, x)$ to the Shield contract for verification.
           
          Using web3: \texttt{nfTokenShield.transfer(proof, inputs, vkId)}
          %remember where the count (enumi) is up to and store it in ongoingEnumCounter:
          \setcounter{ongoingEnumCounter}{\value{enumi}}
        \end{enumerate}
        \ \\
        \midrule
        \textbf{Shield contract's steps:}\\
        \begin{enumerate}
          %resume counter
          \setcounter{enumi}{\value{ongoingEnumCounter}}
          \item Verify the proof as correct: call a Verifier contract to verify the \texttt{(proof, inputs)} pair against the verification key represented by \texttt{vkId}.
          \setcounter{ongoingEnumCounter}{\value{enumi}}
        \end{enumerate}
        \ \\
        \hline
        ... 
			\end{tabular}
		\end{framed}
	\end{center}
\caption{Non-Fungible Transfer Algorithm}
\label{fig:nfTransferAlgorithm}
\end{figure}

%continue on next page
\begin{figure}[htbp]
  \ContinuedFloat %to continue
	\begin{center}
		\begin{framed}
      \begin{tabular}{p{16cm}}
       \textbf{ Verifier contract's steps:}\\
        \begin{enumerate}
          \setcounter{enumi}{\value{ongoingEnumCounter}}
          \item Compute \texttt{result = verify(proof, inputs, vkId)}.
          
          I.e. Verify the \texttt{(proof, inputs)} pair against the verification key.
          \item Return \texttt{result}$\in$\texttt{\{false, true\}} to the Shield contract.
          \setcounter{ongoingEnumCounter}{\value{enumi}}
        \end{enumerate}
        \ \\
        \midrule
        \textbf{Shield contract's steps:}\\
        \begin{enumerate}
          \setcounter{enumi}{\value{ongoingEnumCounter}}
          \item If \texttt{result = false}, revert.
          \item Else:
          \begin{enumerate}
            \item Check $\roott_{n+m-1}$ is in $\rootsList$. (Revert if not).
            \item Check $N_A$ is not already in its list of `spent' nullifiers. (Revert if not).
            \item Add $Z_B$ to the next empty leaf of the Merkle Tree.
            \item Recalculate the path to the root of the Merkle Tree from $Z_B$ for future users.
            \item Append the newly calculated root $\roott_{n+m}$ to the ever-increasing array $\rootsList$
            \item Similarly append the nullifier $N_{A}$ to the ever-increasing array $\bm N$.
          \end{enumerate}
          \setcounter{ongoingEnumCounter}{\value{enumi}}
        \end{enumerate}
        \ \\
        \midrule
        \textbf{Alice's steps:}\\
        \begin{enumerate}
          \setcounter{enumi}{\value{ongoingEnumCounter}}
          \item Store relevant data in her local database, including the leafindex of $Z_B$.
          \item Send Bob important data privately via Whisper (using his public key $pk^W_B$):
          \begin{enumerate}
            \item The salt $\sigma_{\vec{AB}}$ of $Z_B$.
            \item The public key of Bob, $pk^Z_B$, used by Alice in the preimage of $Z_B$ (for completeness, so Bob can check the correctness of $Z_B$ himself).
            \item The tokenId $\alpha$.
            \item $Z_B$.
            \item The leafIndex of $Z_B$ within the on-chain Merkle Tree $M$ (so Bob can locate it).
          \end{enumerate}
          \setcounter{ongoingEnumCounter}{\value{enumi}}
        \end{enumerate}
        \ \\
        \midrule
        \textbf{Bob's steps:}\\
        \begin{enumerate}
          \setcounter{enumi}{\value{ongoingEnumCounter}}
          \item Check the correctness of the information provided by Alice:
          \begin{enumerate}
            \item Check $Z_B$ equals $h(\;\alpha\;|\;pk^Z_B\;|\;\sigma_{\vec{AB}}\;)$
            \item Check that $Z_B$ is stored at the leafIndex of $M$ which Alice claimed.
          \end{enumerate}
          \item Store relevant data in his local database, including whether or not his `correctness checks' passed.
          \setcounter{ongoingEnumCounter}{0} %reset for next figure
        \end{enumerate} 
			\end{tabular}
		\end{framed}
	\end{center}
\caption{Non-Fungible Transfer Algorithm} %same caption as the first part of this figure
%\label{fig:nfTransferAlgorithm} - no label in this second part of the figure
\end{figure}


\newpage
\subsubsection{Details}
\label{sec:721TransferDetails}

We refer to the numbered steps of Figure~\ref{fig:nfTransferAlgorithm}.

\textbf{Step $1$}
\ \\
This is handled at the time Bob creates an account through the \hyperref[sec:ui]{UI}.\\
\\
\textbf{Step $2$}
\ \\
This is handled within the \hyperref[sec:ui]{UI} microservice (or within the api-gateway).\\
\\
\textbf{Step $3$}
\ \\
This is handled within the api-gateway when a call is made by Alice to transfer to Bob.\\
\\
\textbf{Steps $4 - 5$}
\ \\
These steps are handled within \hyperref[sec:nf-token-controller]{\texttt{nf-token-controller.js}}.\\
\\
\textbf{Steps $6 - 7$}
\ \\
These calls to the Shield contract are handled within \hyperref[sec:nf-token-zkp]{\texttt{nf-token-zkp.js}}.\\
\\
It is important at this stage to note that there are an unknown number of other parties utilising the Shield contract.
Hence, the dynamic array of tokens $\bm{Z}$ might have grown since Alice appended her $Z_A$ as the $n^{th}$ leaf of $M$ (during the \hyperref[sec:721Mint]{Mint} explanation).\\
Suppose there have been $m-1$ additional tokens added to $M$ since Alice added $Z_A$.
That is,\\
\begin{align*}
  \bm{Z}_{n+m-1} = (Z_0, Z_1,...,Z_{n-1}, Z_A, Z_{n+1},..., Z_{n+m-1})
\end{align*}
We denote the corresponding Merkle Tree which holds tokens $\bm{Z}_{n+m-1}$ by $M_{n+m-1}$.
We denote its root by $\roott_{n+m-1}$; an element of $\rootsList = (\roott_0, \roott_1,...,\roott_{n+m-1})$.\\
\\
\begin{align*}
  \begin{forest}
    [{$\roott_{n+m-1}:= h\br*{
                        h\br*{
                          h\br*{
                            h\br*{
                              Z_0,Z_1
                            },
                            ...
                          },
                          h\br*{
                            h\br*{
                              Z_{n-1},Z_A
                            },
                            h\br*{
                              Z_{n+1},...
                            }
                          }
                        },
                        h\br*{
                          h\br*{
                            h\br*{
                              Z_{n+m-1},0
                            },
                            0
                          },
                          0
                        }
                      }
                    $}
      [{$ h\br*{
            h\br*{
              h\br*{
                Z_0,Z_1
              },
              ...
            },
            h\br*{
              h\br*{
                Z_{n-1},Z_A
              },
              h\br*{
                Z_{n+1},...
              }
            }
          }
        $}
        [{$ h\br*{
              h\br*{
                Z_0,Z_1
              },
              ...
            }
          $}
          [{$ h\br*{
                Z_0,Z_1
              }
            $}
            [{$Z_0$}][{$Z_1$}]
          ]
          [...
            [...][...]
          ]
        ]
        [{$ h\br*{
              h\br*{
                Z_{n-1},Z_A
              },
              h\br*{
                Z_{n+1},...
              }
            }
          $}
          [{$ h\br*{
                Z_{n-1},Z_A
              }
            $}
            [{$Z_{n-1}$}][{$Z_A$}]
          ]
          [{$ h\br*{
                Z_{n+1},...
              }
            $}
            [$Z_{n+1}$][...]
          ]
        ]
      ]
      [{$ h\br*{
            h\br*{
              h\br*{
                Z_{n+m-1},0
              },
              0
            },
            0
          }
        $}
        [{$ h\br*{
              h\br*{
                Z_{n+m-1},0
              },
              0
            }
          $}
          [{$ h\br*{
                Z_{n+m-1},
                0
              }
            $}
            [{$Z_{n+m-1}$}][0]
          ]
          [0
            [0][0]
          ]
        ]
        [0
          [0
            [0][0]
          ]
          [0
            [0][0]
          ]
        ]
      ]
    ]
  \end{forest}
\end{align*}

\noindent
Alice retrieves the value of the current Merkle root, $\roott_{n+m-1}$, from the Shield contract.\\
\\
Since Alice knows that $Z_A$ is at leaf-index $n$ of $M_{n+m-1}$, Alice can also retrieve the path from the leaf $Z_{n}=Z_A$ to the root $\roott_{n+m-1}$. Path computations are done in \texttt{zkp/src/format-inputs.js}.\\
\\
We denote this path:
\begin{align*}
  \phi_{Z_A} = [\phi_{d-1}, \phi_{d-2},..., \phi_{1}, \phi_0]
\end{align*}
Note that $\phi_0 = \roott_{n+m-1}$.\\
\\
Alice also retrieve's the `sister-path' of this path:
\begin{align*}
  \psi_{Z_A} = [\psi_{d-1}, \psi_{d-2},..., \psi_{1}, \psi_0]
\end{align*}
where $\psi_0 = \phi_0 = \roott_{n+m-1}$\\
\\
For ease of reading, let's focus only on the nodes of $M_{n+m-1}$ which Alice cares about for the purposes of transferring to Bob:

\begin{align*}
  \begin{forest}
    [{$\roott_{n+m-1}:=\phi_0=\psi_0$}
      [{$\phi_1$}
        [{$\psi_2$}
          [...
            [...][...]
          ]
          [...
            [...][...]
          ]
        ]
        [{$\phi_2$}
          [{$\phi_3$}
            [{$\psi_4$}][{$Z_A$}]
          ]
          [{$\psi_3$}
            [...][...]
          ]
        ]
      ]
      [{$\psi_1$}
        [...
          [...
            [...][0]
          ]
          [0
            [0][0]
          ]
        ]
        [0
          [0
            [0][0]
          ]
          [0
            [0][0]
          ]
        ]
      ]
    ]
  \end{forest}
\end{align*}


\noindent
Equipped with $\psi_{Z_A}$, Alice can prove that she owns a token commitment at one of the leaves of $M_{n+m-1}$, without revealing that it is "$Z_n$ located at leaf-index $n$".\\
\\
\textbf{Steps $8 - 9$}
\ \\
These steps are handled within \hyperref[sec:nf-token-controller]{\texttt{nf-token-controller.js}}.\\
\\
As a reminder, we let:
\begin{center}
  \begin{tabular}{l l}
    $x = (N_{A},\
          \roott_{n+m-1},\
          Z_B)$ & Public Inputs used to generate the Proof\\
    $\omega = (\alpha,\
              \psi_{Z_A},\
              sk_A,\
              \sigma_{A},\
              pk_B,\
              \sigma_{\vec{AB}})$ & Private Inputs used to generate the Proof\\
  \end{tabular}
\end{center}
\ \\
\textbf{Steps $10 - 11$}
\ \\
These steps are handled within a \hyperref[sec:zokrates]{ZoKrates} container.\\
\\
Alice uses the $C_{nft-transfer}$ (or $C$) -- the set of constraints for a non-fungible transfer, located in \texttt{zkp/code/gm17/nft-transfer} (see \hyperref[sec:trustedSetup]{Trusted Setup}). $C_{nft-transfer}(\;\omega,\;x\;)$ returns a value of $true$ if Alice provides a set of valid `satisfying' arguments $(\omega, x)$ to $C$.\\
\\
Let's elaborate on each of the checks and calculations constraining the inputs to $C$ (we highlight public inputs in \textbf{bold} below):
\begin{enumerate}
  \item Calculate $h(sk_A) =: pk_A'$.\\
    Note that this newly calculated $pk_A'$ should equal $pk_A$ (Alice's public key), but we don't need to pass $pk_A$ as a private input and explicitly check that $pk_A'=pk_A$; a check on the correctness of $sk_A$ (and hence $pk_A'$) is implicitly achieved in the next two steps:
  \item Calculate $h(\alpha\;|\;pk_A'\;|\;\sigma_A) =: Z_A'$.\\
    Note again that this newly calculated $Z_A'$ should equal $Z_A$ (Alice's token commitment), but we don't need to pass $Z_A$ as a private input and explicitly check that $Z_A'=Z_A$; a check on the correctness of $Z_A$ (and hence $Z_A'$) is implicitly achieved in the next step:
  \item Check inputs $\psi_{Z_A}=[\psi_{d-1}, \psi_{d-2},..., \psi_{1}, \bm{\psi_{0}=\roott_{n+m-1}}]$ and the newly calculated $Z_A'$ satisfy:\\
    $h\br*{\psi_{1}\;|...|\;h\br*{\psi_{d-2}\;|\;h\br*{\psi_{d-1}\;|\;Z_A'}\;}...} = \roott_{n+m-1} ( =: \bm{\psi_{0}})$\\
    Given the one-way nature of our hashing function $h$, the only feasible way we could have arrived at the correct value of $\roott_{n+m-1}$ is if the sister-path $\psi_{Z_A}$ is correct, and if $Z_A'$ is correct, which (working backwards) must mean that $sk_A$ is correct.

    How does the circuit know the value of $\roott_{n+m-1}$ is correct? It doesn't; but it is a `public input', and we can rely upon the Shield smart contract to check the correctness of all public inputs.\\
  \\
  We've therefore shown in the steps so far, that:
  \begin{itemize}
    \item[--] Alice is the owner of a token commitment (because she knows its secret key)
    \item[--] Said token commitment is indeed a leaf of the on-chain Merkle Tree $M_{n+m-1}$.
  \end{itemize}

  Alice commits to spending her token $Z_A$ in the next step:
  \item Check inputs $\sigma_{A}, sk_A, \bm{N_A}$ satisfy:
    $h(\sigma_{A}\;|\;sk_A) = \bm{N_{A}}$\\
    $N_A$ is referred to as a `nullifier' because it is understood by all participants to be an indisputable commitment to spend (`nullify') a token commitment. Remember that the token commitment being spent isn't revealed; the earlier steps have allowed Alice to demonstrate hidden knowledge of the secret key $sk_A$ of a token commitment which does indeed exist. By including $sk_A$ in the nullifier's preimage, Alice is binding herself as the executor of this transfer. By including $\sigma_A$, Alice is specifying a serial number which is unique to the token $Z_A$ (thereby distinguishing this nullifier from those which would nullify any other token commitments she may own).\\
  \item Check inputs $\alpha, pk_B, \sigma_{\vec{AB}}, \bm{Z_B}$ satisfy:
    $h(\alpha\;|\;pk_B\;|\;\sigma_{\vec{AB}}) = \bm{Z_B}$\\
    This final step constrains the same asset $\alpha$ to be included in $Z_B$ as was included in $Z_A$.\\
    You might notice that the circuit doesn't actually constrain Alice to use the correct values for Bob's public key $pk_B$, nor the serial number $\sigma_{\vec{AB}}$ as inputs to the circuit. Alice is free to transfer ownership of the token commitment to anyone.
\end{enumerate}
Notice how each stage is linked to the last, and that at each of the `Check' stages, private inputs are being reconciled against at least one public input (highlighted in \textbf{bold} to help you notice). By structuring the circuit $C$ in this way, we are able to share only the public inputs with the Shield contract (along with a `proof' $\pi_{C,x,\omega}$). We'll see shortly that the Shield contract checks the correctness of each of the public inputs against its current states.\\
\\

\noindent
If all of the above constraints are satisfied by the public and private inputs, ZoKrates will generate the proof $\pi_{C,x,\omega}$; a proof of knowledge of satisfying arguments $(\omega, x) \ s.t. \ C(\omega, x) = 1$.\\
\\
\textbf{Step $12$}
\ \\
This transaction is handled within \hyperref[sec:nf-token-zkp]{\texttt{nf-token-zkp.js}}.\\
\\
Having generated $\pi_{C,x,\omega}$, Alice then sends the following to the Shield contract from her anonymous Ethereum address $\Xi_{A,1}$:
\begin{align*}
  &\pi_{C,x,\omega}\\
  &x = (N_{A}, \roott_{n+m-1}, Z_B)
\end{align*}
\\
Recall that everyone knows the checks and calculations which have been performed in the circuit $C_{nft-transfer}$, because it is a public file in the Nightfall repository. Further, everyone knows the verification key $vk_C$ which uniquely represents this circuit, because it has been publicly stored in the Verifier Registry contract. Therefore, when this anonymous caller (Alice) shares the pair $(x, \pi_{C,x,\omega})$, and the `unique id' of the relevant verification key $vk_C$; everyone will interpret this information as the caller's intention to transfer, and everyone will be convinced that the caller knows the secret key which permits them to transfer ownership of a token commitment.\\
\\



\textbf{Steps $13 - 15$}
\ \\
The Verifier Registry contract already has stored within it the verification key $vk_C$.
It runs a verification function $V(vk_C, \pi_{C,x,\omega}, x)$.
\begin{align*}
  V: (vk_C, \pi_{C,x,\omega}, x) \to \{0,1\}
\end{align*}
where:
\[
    V=
\begin{cases}
    1,& \text{if } \pi_{C,x,\omega} \text{ and } x \text{ satisfy } vk_C\\
    0,& \text{otherwise}
\end{cases}
\]
\ \\


\textbf{Steps $16 - 17$}
\ \\
If the Verifier contract returns $1$ ($true$) (verified) to the Shield contract, then the Shield contract will be satisfied that Alice's proof and public inputs represent her commitment to relinquish ownership of a token commitment, and to transfer ownership of the underlying asset to someone via the newly proposed token commitment $Z_B$. If the Verifier contract returns $0$, then the transaction will revert.\\
\\
Let's suppose Alice's $(x, \pi_{C,x,\omega})$ pair is verified.\\
\\
Following verification of the proof, the Shield contract will do the following:
\begin{enumerate}
  \item Check $\roott_{n+m-1}$ is in $\rootsList$.\\
    (If not, the transfer will fail)
  \item Check $N_A$ is not already in the list of nullifiers, which we denote $\bm{N}$.\\
    (If $N_A$ is already in $\bm{N}$, the transfer will fail)
  \item Append the commitment $Z_B$ to the ever-increasing array of tokens, $\bm{Z}_{n+m}$, so that $\bm{Z}_{n+m}=(Z_0, Z_1, ... Z_{n-1}, Z_A, Z_{n+1}, ... Z_{n+m-1}, Z_B)$\\
  \item Recalculate a Merkle Root $\roott_{n+m}$ of $M_{n+m}$\\
    \\
    \begin{align*}
      \begin{forest}
        [{$\roott_{n+m}:= h\br*{
                            h\br*{
                              h\br*{
                                h\br*{
                                  Z_0,Z_1
                                },
                                ...
                              },
                              h\br*{
                                h\br*{
                                  Z_{n-1},Z_A
                                },
                                h\br*{
                                  Z_{n+1},...
                                }
                              }
                            },
                            h\br*{
                              h\br*{
                                h\br*{
                                  Z_{n+m-1}, Z_B
                                },
                                0
                              },
                              0
                            }
                          }
                        $}
          [{$ h\br*{
                h\br*{
                  h\br*{
                    Z_0,Z_1
                  },
                  ...
                },
                h\br*{
                  h\br*{
                    Z_{n-1},Z_A
                  },
                  h\br*{
                    Z_{n+1},...
                  }
                }
              }
            $}
            [{$ h\br*{
                  h\br*{
                    Z_0,Z_1
                  },
                  ...
                }
              $}
              [{$ h\br*{
                    Z_0,Z_1
                  }
                $}
                [{$Z_0$}][{$Z_1$}]
              ]
              [...
                [...][...]
              ]
            ]
            [{$ h\br*{
                  h\br*{
                    Z_{n-1},Z_A
                  },
                  h\br*{
                    Z_{n+1},...
                  }
                }
              $}
              [{$ h\br*{
                    Z_{n-1},Z_A
                  }
                $}
                [{$Z_{n-1}$}][{$Z_A$}]
              ]
              [{$ h\br*{
                    Z_{n+1},...
                  }
                $}
                [$Z_{n+1}$][...]
              ]
            ]
          ]
          [{$ h\br*{
                h\br*{
                  h\br*{
                    Z_{n+m-1}, Z_B
                  },
                  0
                },
                0
              }
            $}
            [{$ h\br*{
                  h\br*{
                    Z_{n+m-1}, Z_B
                  },
                  0
                }
              $}
              [{$ h\br*{
                    Z_{n+m-1}, Z_B
                  }
                $}
                [{$Z_{n+m-1}$}][{$Z_B$}]
              ]
              [0
                [0][0]
              ]
            ]
            [0
              [0
                [0][0]
              ]
              [0
                [0][0]
              ]
            ]
          ]
        ]
      \end{forest}
    \end{align*}

    Note that the Shield contract only needs to calculate the hashes on the path from $Z_B$ to the root.

  \item Append $\roott_{n+m}$ to the ever-increasing array $\rootsList$
  \item Similarly append the nullifier $N_{A}$ to the ever-increasing array $\bm N$.
\end{enumerate}
\ \\

\textbf{Steps $18 - 19$}
\ \\
The api-gateway routes the data resulting from a transfer to her local database.\\
\\
Similarly, the api-gateway ensures any sensitive data (data which is private to Alice alone) is filtered before Alice sends data to Bob.\\
\\
Data which is crucial to Bob verifying his ownership of the new $Z_B$ is encrypted with Bob's public whisper key $pk^W_B$ and broadcast to the Whisper network.\\
\\

\textbf{Steps $20 - 21$}
\ \\
Nightfall uses web3.shh to use Whisper. Bob's logged-in application will listen for all Whisper messages, and will try to decrypt all messages with his private whisper key $sk^W_B$. If decryption is successful, the data will be stored in the relevant database on Bob's local machine.\\
\\
\hyperref[sec:nf-token-zkp]{\texttt{nft-token-zkp.js}} includes functions to cross-reference the data Bob has received from Alice against the data stored in the Shield contract.\\
\\
Bob will store all important information in his private database.

